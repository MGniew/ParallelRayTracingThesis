\section{Szczegółowy opis wybranych fragmentów kodu}
	\subsection{RayTracer}

Klasa \emph{RayTracer} implementuje zestaw metod realizujących algorytm śledzenia promieni. Korzysta ona z interfejsu udostępnianego przez klasy takie jak \emph{Scene} czy \emph{Camera}, aby generować kolejne promienie do wysłania. Poniżej zostały omówione dwa najważniejsze fragmenty kodu zawarte w tej klasie:
	
\begin{lstlisting}[caption={Fragment klasy \emph{RayTracer}}]

void RayTracer::recursiveRayTracer(int depth) {

Vector3<float> worldPosOfPixel;
Vector3<float> directionVector;

for(int i = 0; i < scene->getWidth(); i++) {
    for(int j = 0; j < scene->getHeight(); j++) {
        worldPosOfPixel = camera->getWorldPosOfPixel(i + scene->getStartX(),j + scene->getStartY());
        directionVector = worldPosOfPixel - *camera->getEye();
        directionVector.normalize();
        scene->setPixelColor(i, j, getColorRecursive(worldPosOfPixel, directionVector, depth));
    }
}
}
\end{lstlisting}	

Powyższy fragment kodu implementuje algorytm rekursywnej metody śledzenia promieni. Dla każdego piksela sceny zostaje wygenerowany promień pierwotny (\emph{directionVector}), który następnie jest wysyłany w scenę - funkcja \emph{getColorRecursive} (opisana niżej) zwraca kolor, jaki należy przypisać danemu punktowi ekranu. Każdy z węzłów wykonawczych posiada swój egzemplarz obiektu klasy \emph{Scene} (wykorzystywany w powyższym kodzie) zmodyfikowany w taki sposób, aby przechowywał on jedynie fragment obrazu (\emph{Pixels)} - początek wycinka jest określany zmiennymi \emph{startX} i \emph{startY}, a zmodyfikowane wymiary obrazu pozwalają określić jego koniec. Więcej o komunikacji Master/Slave można przeczytać w rozdziale czwartym. 

\begin{lstlisting}[caption={Fragment klasy \emph{RayTracer}}]

Vector3<float> RayTracer::getColorRecursive(Vector3<float> startPoint, Vector3<float> directionVector, int depth)
{

SceneObject* sceneObject;
Vector3<float> crossPoint;
Vector3<float> reflectedRay;
Vector3<float> localColor;
Vector3<float> reflectedColor;

//refraction
Vector3<float> transparencyColor;
Vector3<float> transparencyRay;

if (depth == 0)
    return Vector3<float>();

depth--;

sceneObject = scene->getClosest(crossPoint, startPoint, directionVector);

if (sceneObject == nullptr)
    return Vector3<float>(*scene->backgroundColor);

Vector3<float> normalVector = sceneObject->getNormalVector(crossPoint);
Vector3<float> observationVector = directionVector*-1;


if (observationVector.scalarProduct(normalVector) < 0) {
    normalVector = normalVector*-1;
}

if(sceneObject->getTransparency()>0) {
    transparencyRay = directionVector.refract(normalVector, sceneObject->getDensity(), 1);
    transparencyColor = getColorRecursive(crossPoint, transparencyRay, depth);
}

if (sceneObject->getLocal()>0) {
    localColor.setValues(sceneObject->getLocalColor(normalVector, crossPoint, observationVector));
}

if (sceneObject->getMirror()>0) {
    reflectedRay = directionVector.reflect(normalVector);
    reflectedColor = getColorRecursive(crossPoint, reflectedRay, depth);
}

return localColor*sceneObject->getLocal() + reflectedColor*sceneObject->getMirror() + transparencyColor*sceneObject->getTransparency();

}

\end{lstlisting}
	
Powyższa funkcja jest rekurencyjnie wywoływaną metodą pozwalającą określić ostateczny kolor piksela, z którego został wysłany promień pierwotny (wysłanie promienia pierwotnego następuje w funkcji \emph{recursiveRayTracer}). Przyjmuje ona promień (w postaci punktu początkowego i wektora kierunku) oraz zmienną określającą głębokość drzewa - jest ona dekrementowana z każdym kolejnym rekurencyjnym wywołaniem funkcji, a kiedy osiągnie zero, rekurencja jest przerywana. Pierwszym krokiem algorytmu jest stwierdzenie, czy promień przeciął się z jakimś obiektem (jest tutaj wykorzystywany albo przegląd zupełny, albo drzewo BSP). Jeżeli nie, to ostateczny kolor piksela (lub jego składowa na danym poziomie drzewa rekurencji) przyjmuje wartość koloru tła. Jeżeli tak, to algorytm wybiera obiekt będący najbliżej początku promienia (obiekt widoczny z perspektywy tego punktu) i (w zależności od modelu \emph{Phonga} i parametrów powierzchni omówionych w rozdziale drugim) ustala lokalną barwę obiektu oraz wysyła dwa kolejne promienie mające wpływ na barwę ostateczną - promień odbity od powierzchni i promień przez nią przechodzący (jest tutaj uwzględnianie złamanie światła).
Ostateczny kolor piksela jest sumą kolorów lokalnych osiągniętych przez wszystkie promienie powstałe w wyniku wysłania promienia pierwotnego.	Niezrozumiały może wydawać się następujący fragment:

\begin{lstlisting}[caption={}]
if (observationVector.scalarProduct(normalVector) < 0) {
    normalVector = normalVector*-1;
}
\end{lstlisting}

Biorąc pod uwagę, że kierunek wektora normalnego ma wpływ na otrzymany kolor lokalny powierzchni (jeżeli jego kierunek jest niezgodny z kierunkiem światła to znaczy, że powierzchnia nie jest oświetlona), należy go odwrócić tak, aby był on zgodny z kierunkiem obserwacji - np. w sytuacji, w której obserwator znajdowałby się w kuli (wraz ze światłem oświetlającym scenę), a wektor normalny do powierzchni kuli skierowany byłby na zewnątrz, oświetlenie i tak nie miałoby na nią wpływu.

\subsection{BSP}

W tym punkcie zostanie przedstawione w jaki sposób zaimplementowano budowę drzewa BSP oraz jego przeglądanie.

\begin{lstlisting}[caption={Fragment klasy \emph{BSP} - budowa drzewa}]

root->partitionPlane = getBestPlane(polygons);
while(!polygons.empty()) {
    object = polygons.back();
    polygons.pop_back();
    result = root->partitionPlane.classifyObject(object);
    switch (result) {
        case FRONT:
            frontList.push_back(object);
        break;

        case BACK:
            backList.push_back(object);
        break;

        case COINCIDENT:
            backList.push_back(object);
            frontList.push_back(object);
        break;

        case SPANNING: {
            if (object->getType() == 's') {
                root->polygons.push_back(object);
            } else {
                Triangle *triangle = static_cast<Triangle*>(object);
                std::list<Triangle*> tempFrontList, tempBackList;
                triangle->split(root->partitionPlane, tempFrontList, tempBackList);
                while (!tempBackList.empty()) {
                    backList.push_back(tempBackList.back());
                    tempBackList.pop_back();
                }
                while (!tempFrontList.empty()) {
                    frontList.push_back(tempFrontList.back());
                    tempFrontList.pop_back();
            }
            }
        }
        break;

        default:
            break;
        }
    }
\end{lstlisting}

Powyższy fragment jest fragmentem kodu funkcji budującej drzewo opisane w punkcie 2.2.4. Pierwszym krokiem każdej kolejnej rekurencji budowy drzewa jest ustalenie płaszczyzny podziału - brane są pod uwagę wszystkie te, które są wyznaczane przez trójkąty zawarte w danym wierzchołku i te, które są do tych trójkątów prostopadłe (styczne z krawędziami). Poprzez najlepszą płaszczyznę rozumie się taką, która dzieli trójkąty na równe ilościowo grupy. Następnie w pętli algorytm sprawdza, po której stronie wybranej płaszczyzny znajduje się dany obiekt z listy - w zależności od sytuacji trafia on do listy, która zostanie przekazana kolejnym dzieciom (,,przedniemu'' i ,,tylnemu''). W przypadku, gdy trójkąt leży na płaszczyźnie podziału, jest on umieszczany w obu listach, z kolei jeżeli płaszczyzna go przecina, to jest on dzielony na dwa (w sytuacji powstania trójkąta i czworokąta, czworokąt jest dzielony na dwa trójkąty) - każda z połówek trafi do odpowiedniego ,,dziecka''. 	Program uwzględnia sfery przechowywane w postaci równania (a więc prosty podział takiego obiektu nie jest możliwy). W momencie, w którym płaszczyzna przetnie sferę, trafia ona tylko do listy obiektów rozpatrywanego wierzchołka. Zaimplementowanie sfer wymagało takiej niekonwencjonalnej modyfikacji algorytmu, która będzie miała wpływ na sposób przeglądania drzewa. Rekurencja kończy się, kiedy zostanie osiągnięta maksymalna głębokość drzewa (określana przez parametr przekazywany do funkcji przy pierwszym wywołaniu), lub w sytuacji, w której dalszy podział jest niemożliwy. Wtedy wierzchołek jest zamieniany w liść (wskaźniki na potomstwo są puste), a wejściowa lista obiektów zostaje do niego przypisana. Nie biorąc pod uwagę sfer wszystkie wierzchołki niebędące liśćmi są puste.

\begin{lstlisting}[caption={Fragment klasy \emph{BSP} - przeglądanie drzewa}]

SceneObject *BSP::intersect(BSP::node *root, Vector3<float> &crossPoint, Vector3<float> &startingPoint, Vector3<float> &directionVector) {

if (root->back == nullptr && root->front == nullptr) {
    return getClosestInNode(root->polygons, crossPoint, startingPoint, directionVector);
}

SceneObject *thisNodeHit = nullptr;
Vector3<float> tempCross;

if(!root->polygons.empty()) {
    thisNodeHit = getClosestInNode(root->polygons, tempCross, startingPoint, directionVector);
}

node *near;
node *far;
SceneObject *hit = nullptr;;

switch (root->partitionPlane.classifyPoint(&startingPoint)) {
    case FRONT:
        near = root->front;
        far = root->back;
    break;

    case BACK:
        near = root->back;
        far = root->front;
    break;

    case COINCIDENT: {
        Vector3<float> point = startingPoint + directionVector;
        if (root->partitionPlane.classifyPoint(&point) == FRONT) {
            near = root->front;
            far = root->back;
        }
        else {
            near = root->back;
            far = root->front;
        }
    }
    break;

    default:
        return nullptr;
        break;
}

hit = intersect(near, crossPoint, startingPoint, directionVector);

if (hit == nullptr && root->partitionPlane.rayIntersectPlane(startingPoint,directionVector)) {
    hit = intersect(far, crossPoint, startingPoint, directionVector);
}

if (thisNodeHit != nullptr) {
    if (hit != nullptr) {
        if (tempCross.distanceFrom(startingPoint) < crossPoint.distanceFrom(startingPoint)) {
            hit = thisNodeHit;
            crossPoint = tempCross;
        }
    }
    else {
        hit = thisNodeHit;
        crossPoint = tempCross;
    }
}
return hit;
}

\end{lstlisting}

Powyżej przedstawiono rekurencyjny algorytm przeszukiwania drzewa. Funkcja ta przyjmuje wskaźnik na sprawdzany wierzchołek drzewa i promień, a zwraca wskaźnik na znaleziony obiekt i punkt przecięcia (zmienna \emph{crossPoint} widoczna w liście parametrów).

Pierwszym krokiem algorytmu jest sprawdzenie, czy dany wierzchołek jest liściem. Jeżeli tak, to zostaje przeprowadzony test przecięcia na każdym obiekcie znajdującym się w wierzchołku - wybieramy najbliższy trafiony i zwracamy go do funkcji wywołującej. Następnie należy sprawdzić, czy dany wierzchołek rzeczywiście jest pusty (komplikacje spowodowane nietypowym obiektem nie będącym wielokątem - sferą). Jeżeli nie jest, to ponownie zostaną przeprowadzone testy przecięć dla każdego obiektu znajdującego się w wierzchołku - znaleziony obiekt przechowujemy w zmiennej \emph{thisNodeHit}.

Następnie, w zależności od tego, po której części płaszczyzny znajduje się punkt początkowy promienia, zostają ustawione zmienne ,,near'' (połowa, w której znajduje się punkt początkowy) i ,,far'' (alternatywa). W przypadku, w którym punkt startowy zawiera się w płaszczyźnie dzielącej, sprawdzamy, czy wektor kierunku nie jest równoległy do płaszczyzny - jeżeli jest ,,near'' i ,,far'' nie ma znaczenia; jeżeli nie jest, to jako ,,near'' wybieramy tę połowę wskazywaną przez wektor.

Kolejnym krokiem jest rekurencyjne wywołanie tej funkcji dla potomka ,,near''. Jeżeli nie zwróci ona żadnego obiektu, a promień przecina płaszczyznę dzielącą to rozwiązanie może znajdować się jeszcze w drugim potomku (,,far''). Ostatni fragment kodu sprawdza, czy jeżeli znaleziono obiekt w danym węźle i obiekt w jednym z dzieci, to który z nich jest bliżej - ten zostanie zwrócony jako wynik.

\subsection{MasterThread}
	
	
\begin{lstlisting}[caption={Fragment klasy \emph{MasterThread}}]

void MasterThread::run() {

while (true) {

    splitToChunks(numChunks);

    pending = 0;
    for (int i=1; i<worldSize; i++) {
        if (!sendNextChunk(i)) break;
        pending++;
    }

    int dest;
    while(pending>0) {
        switch(recvMessage()) {
            case EXIT: return; break;
            case PIXELS:
                dest = recvPixels(status);
                if (!sendNextChunk(dest))
                    pending--;
                break;
            default: break;
        }
    }
 	emit workIsReady();
    camera->rotate();
    sendCameraPointToPoint();
}
}

\end{lstlisting}


Powyższy kod przedstawia główną pętlę programu węzła zarządzającego. Zanim zostanie ona wywołana, \emph{MasterThread} rozsyła informacje o wczytanej scenie, kamerze i głębokości drzewa śledzenia promieni do każdego z węzłów wykorzystując komunikację typu \emph{broadcast} (dzieje się to w konstruktorze klasy).

Zgodnie z założeniami rozpoczyna się ona od podziału generowanego obrazu na fragmenty, których definicje trafiają na kolejkę oczekujących zadań. Następnie algorytm zdejmuje zadania z kolejki i wysyła je do kolejnych węzłów wykonawczych (zostaje tutaj ustalona liczba zadań aktualnie wykonywanych). W momencie, w którym master otrzyma wyniki działań (tablicę pikseli) któregokolwiek węzła, następuje sprawdzenie, czy istnieją jeszcze jakieś zadania do wykonania. Jeżeli tak, to jedno z nich zostanie wysłane do węzła, od którego otrzymaliśmy właśnie fragment obrazu, jeżeli nie, to liczba aktualnie wykonywanych zadań zmniejsza się.

W chwili, w której liczba wykonywanych zadań spadnie do zera, zostaje wysłany sygnał do wątku GUI informujący go o tym, że generowana klatka animacji jest gotowa. Zostaje również uaktualniona pozycja kamery, która następnie jest wysyłana do każdego z węzłów wykonawczych.

\subsection{SlaveMPI}

\begin{lstlisting}[caption={Fragment klasy \emph{SlaveMPI}}]

int SlaveMPI::exec() {

RayTracer rayTracer;
while(true) {

    switch(recvMessage()) {
        case EXIT:
            return EXIT; break;
        case CHUNK:
            recvChunk();
            rayTracer.recursiveRayTracer(depth);
            sendPixels(); break;
        case CAMERA:
            recvCameraPointToPoint(); break;
        default: break;
    }
}
return 0;
}

\end{lstlisting}


Powyższa metoda pokazuje, jak działa pętla główna węzłów wykonawczych. Zanim zostanie ona uruchomiona, wszelkie niezbędne informacje dot. sceny zostają odebrane przez każdy z węzłów (dzieje się to w konstruktorze). 

Zgodnie z opisem w rozdziale czwartym program czeka na polecenia mogące nadejść od węzła zarządzającego - może być to żądanie zakończenia programu, definicja fragmentu obrazu, który należy wyznaczyć, czy definicja kamery (jest ona aktualizowana co klatkę animacji). Do wyznaczenia tablicy pikseli program wykorzystuje klasę \emph{RayTracer}, która dokładniej została opisana wyżej. 

\section{Serializacja}

\begin{lstlisting}[caption={Interfejs \emph{Serializable}}]

#ifndef SERIALIZABLE_H
#define SERIALIZABLE_H

#include "vector"

class Serializable
{
public:

    virtual void serialize(std::vector<char> *bytes) = 0;
    virtual void deserialize(const std::vector<char>& bytes) = 0;
    virtual char getType() = 0;
    virtual ~Serializable();
    int serializedSize;

};

#endif // SERIALIZABLE_H

\end{lstlisting}

Kluczowym elementem programu jest potrzeba wysyłania obiektów pomiędzy węzłami. W tym celu potrzebny jest zestaw metod umożliwiających ich serializację i deserializację do strumienia bajtów (tak aby było możliwe wysyłanie obiektów funkcjami udostępnianymi przez \emph{MPI}). Jest on zdefiniowany poprzez interfejs \emph{Serializable}, który wymusza w klasach dziedziczących zdefiniowanie metod \emph{serialize} (serializacja obiektu do wektora bajtów), \emph{deserialize} (deserializacja obiektu ze strumienia bajtów) i \emph{getType} (metoda umożliwiająca ustalenie z jakim typem obiektu mamy do czynienia).