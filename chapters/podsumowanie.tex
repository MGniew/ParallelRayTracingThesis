Podstawowym celem pracy było zbadanie czy metoda śledzenia promieni ma rację bytu w interaktywnych aplikacjach graficznych. By uzyskać odpowiedź na to pytanie przeprowadzono szereg testów pokazujących między innymi, w jaki sposób definicja sceny, której dwuwymiarowa wizualizacja ma się pojawić przed oczami użytkownika, wpływa na czas generowania obrazu. Dzięki nim dowiedzieliśmy się, w jaki sposób czynniki takie jak rozmiar obrazu, liczba świateł, głębokość drzewa wpływają na szybkość obliczeń. Taka wiedza pozwala na umiejętne minimalizowanie czasu potrzebnego na generowanie realistycznych grafik, przy jak niewielkim wpływie na jakość obrazu.

Przejdźmy do omówienia wyników zrównoleglenia algorytmu. Dzięki niemu możliwe było uzyskanie ok. 16 klatek na sekundę przy najmniej skomplikowanej scenie (\emph{Spheres}), która mimo swojej prostoty i tak była bardzo efektowna. Dla większej liczby obiektów znajdujących się na scenie rezultaty nie były wystarczające (mimo maksymalnego przyspieszenia sięgającego 900\%) - klatka dla sceny \emph{Glass} generowała się ok. 0,3 sekundy, dla \emph{Suzanne} 37 sekund, a dla \emph{Kid} ok. 5 min. Takie czasy są niedopuszczalne. W ramach badań nad możliwością przyspieszenia obliczeń generowania obrazu zostało przetestowane drzewo BSP. Okazuje się, że pozwala ono w niektórych sytuacjach (dokładnie w jakich zostało opisane w punkcie 7.1.5) na znaczne przyspieszenie obliczeń (czas generowania obrazu na podstawie sceny \emph{Suzanne} spadł do 8 sekund), jednak, jak pokazuje alternatywny, ,,zły'' przykład \emph{Glass}, potrafi również pogorszyć wydajność. Walką z tego typu sytuacjami może być zmiana strategi podziału przestrzeni na \emph{SAH} (punkt 2.2.4), lub zastosowanie brył otaczających. Istnieje możliwość podejścia hybrydowego, które wykorzystywałoby zalety zarówno drzew BSP jak i drzew BVH - w celu implementacji takiego podejścia należy dokładnie zbadać zachowanie drzew BVH, a następnie próbować łączyć ze sobą oba rozwiązania (prawdopodobnie modyfikując książkowe zachowanie jednego i drugiego). Propozycją, która nie jest uzasadniona żadnymi badaniami a czystą intuicją, jest zamykanie integralnych, nieruchomych (drzewa BVH nie potrzebują przebudowy w przypadku przemieszczenia się obiektów) modeli w bryłach otaczających, a następnie podział tych brył drzewem BSP - takie rozwiązanie powinno zmniejszyć negatywny wpływ wad drzewa BSP.

Czy metoda śledzenia promieni może być wykorzystywana w aplikacjach interaktywnych? Uważam, że tak - badania jakie zostały przeprowadzone pozwalają przypuszczać, że gdyby do zrównoleglenia obliczeń był wykorzystywany koprocesor graficzny (\emph{GPU}), to stopień przyspieszania powinien być wystarczający do budowania np. prostych gier komputerowych opartych na tym algorytmie. Możliwość dalszych optymalizacji, które powinny poprawić rezultaty, zdaje się potwierdzać tę hipotezę. Ciekawa wydaje się również opcja zbudowania klastra obliczeniowego opartego o karty graficzne. Niniejsza praca to dopiero szczyt góry lodowej - ogrom zagadnień i zależności, które należy zgłębić jest przytłaczający i zarazem fascynujący. 