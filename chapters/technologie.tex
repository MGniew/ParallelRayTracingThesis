W tym rozdziale przedstawiono technologie (wraz z uzasadnieniem wyboru) jakie zostały użyte do zaimplementowania programu, którego dotyczy praca.

\section{C++}

Język C++ jest ustandaryzowanym językiem programowania ogólnego przeznaczenia, który został zaprojektowany przez Bjarne Stroustrupa. Umożliwia on stosowanie kilku paradygmatów programowania, w tym programowania obiektowego, które, w przypadku śledzenia promieni, jest rozwiązaniem wskazanym. Programowanie obiektowe, w którym program definiuje się za pomocą obiektów, pasuje do problematyki problemu (program składać się będzie ze sceny, jej elementów, kamery itd.). Mechanizmy abstrakcji takie jak dziedziczenie, enkapsulacja, czy polimorfizm pozwolą na wygodne zaprogramowanie obsługi różnego typu obiektów sceny.

Dodatkowo język C++ słynie z wydajności i pozwala na bezpośrednie zarządzanie pamięcią - te właściwości pozwalają na napisanie zoptymalizowanego (pod względem czasu wykonania i zużycia pamięci) programu, co jest kluczowym elementem tematu niniejszej pracy. 

\section{Qt}

Qt jest zestawem bibliotek i narzędzi do tworzenia graficznego interfejsu użytkownika w językach takich jak C++, Java, QML, C\#, Python i wielu innych. Qt zapewnia mechanizm sygnałów i slotów, automatyczne rozmieszczanie widżetów i system obsługi zdarzeń. Środowisko jest dostępne między innymi dla systemów Windows, Linux, Solaris, Symbian i Android. Popularność rozwiązania, elastyczność, duża społeczność i wsparcie ze strony producenta \cite{qt} sprawiają, że Qt jest dobrym wyborem przy pisaniu aplikacji okienkowych.


\section{Standard MPI}

Wybór sposobu zrównoleglenia jest podyktowany nie tylko rodzajem problemu, którego dotyczy praca, ale również rodzaju dostępnego sprzętu. Najbardziej elastyczną technologią pozwalającą na obliczenia równoległe są klastry - grupa połączonych ze sobą niezależnych komputerów mogących różnić się podzespołami. Minusem takiego rozwiązania jest to, że w przeciwieństwie do systemów wieloprocesorowych, procesory nie są podłączone magistralą ze wspólną pamięcią, co z kolei oznacza wolniejszą i trudniejszą programistycznie komunikację między nimi. W taki, alternatywny sposób, wiele problemów mogłoby być rozwiązane efektywniej. Kolejnym problemem jest trudność rozłożenia obliczeń pomiędzy stacjami wykonawczymi, ponieważ czas obliczeń (i czas przesyłu danych przez sieć) może być znacząco różny dla poszczególnych komputerów. W metodzie śledzenia promieni narzut komunikacyjny jest relatywnie niski, a sugerowany w punkcie 2.1.3 sposób zrównoleglenia obliczeń nie powinien stanowić dużego problemu w ich rozłożeniu, więc klaster obliczeniowy jest dobrym rozwiązaniem, zwłaszcza że jest to rozwiązania tanie, dostępne i łatwe w rozbudowie. Pomijając dodawanie nowych węzłów, stacje nie muszą ograniczać się do jednego rodzaju podzespołów - wykorzystując koprocesory takie jak ,,Xeon Phi'', różnego rodzaju karty graficzne, FPGA, czy inne dedykowane układy, można zyskać znaczną moc obliczeniową, ale (tak jak to jest napisane wyżej) nie każdy zrównoleglalny problem będzie efektywnie rozwiązywany taką technologią \cite{wikiPar}. 

MPI (Message Passing Interface) jest standardem przesyłania komunikatów pomiędzy procesami znajdującymi się na jednym lub wielu komputerach. Standard ten operuje na na architekturze MIMD (Multiple Instructions Multiple Data) - każdy proces wykonuje się we własnej przestrzeni adresowej, pracuje na różnych danych i może wykonywać różne instrukcje. MPI udostępnia bogaty interfejs pozwalający zarówno na komunikację typu punkt - punkt, jak i komunikację zbiorową. Jedną z implementacji standardu jest MPICH. Na stronie producenta można znaleźć bogatą dokumentację i poradniki dot. tej technologii \cite{mpich}. 